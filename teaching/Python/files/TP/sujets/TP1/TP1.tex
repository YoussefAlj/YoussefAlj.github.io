\documentclass[10pt,a4paper]{article}
\usepackage[utf8]{inputenc}
\usepackage[francais]{babel}
\usepackage[T1]{fontenc}
\usepackage{amsmath}
\usepackage{amsfonts}
\usepackage{amssymb}
\usepackage[top=2cm, left=2cm, right=2cm, bottom=2cm]{geometry}
\usepackage{url}
\usepackage{palatino, mathpazo}

\usepackage{listings}
\usepackage{xcolor}

\definecolor{codegreen}{rgb}{0,0.6,0}
\definecolor{codegray}{rgb}{0.5,0.5,0.5}
\definecolor{codepurple}{rgb}{0.58,0,0.82}
\definecolor{backcolour}{rgb}{0.95,0.95,0.92}

\lstdefinestyle{mystyle}{
    backgroundcolor=\color{backcolour},   
    commentstyle=\color{codegreen},
    keywordstyle=\color{magenta},
    numberstyle=\tiny\color{codegray},
    stringstyle=\color{codepurple},
    basicstyle=\ttfamily\footnotesize,
    breakatwhitespace=false,         
    breaklines=true,                 
    captionpos=b,                    
    keepspaces=true,                 
    numbers=left,                    
    numbersep=5pt,                  
    showspaces=false,                
    showstringspaces=false,
    showtabs=false,                  
    tabsize=4,
    language=Python
}

\lstset{style=mystyle}

\title{Langage Python\\
TD-TP 1} 
\author{Y. \bsc{Alj}}
\date{}
\begin{document}
\maketitle

\section{Pré-requis}
\begin{itemize}
\item Connaître l'architecture de votre processeur (32 ou 64 bits).
\item Installation d'Ananconda  : \url{https://www.anaconda.com/products/individual}. Choisir la version correspondant à l'architecture de votre processeur.
\end{itemize}

\section*{Exercice 1:}
\begin{enumerate}
\item Sans exécuter le script ci-dessous, quels sont les résultats des instructions suivantes.
\item Créer un nouveau script dans Spyder, y coller le script ci-dessous. Les résultats affichés correspondent-ils à vos attentes?
\end{enumerate}


\begin{minipage}{0.5\textwidth}
\lstinputlisting{./progs/exercice1.py}
\end{minipage}



\section*{Exercice 2:}
Ouvrir Spyder. Créer un nouveau fichier et écrire un script Python qui affiche Bonjour. Tester votre script.

\section*{Exercice 3:}
Ecrire un script qui demande deux nombres entiers à l'utilisateur, effectue les opérations ($+$,$-$, $\times$ et $\div$) et affiche les résultats de chacune de ces opérations.

\newpage
\section*{Exercice 4:}
\begin{enumerate}
\item Ecrire un script qui demande à l'utilisateur de saisir un nombre entier $N$, calcule la somme des entiers positifs qui sont inférieurs ou égaux à $N$ et qui affiche le résultat de la somme.
\item Comment pouvez-vous vérifier l'exactitude du résultat trouvé?
\item Modifier le programme précédent pour ne calculer que la somme des nombres pairs.
\end{enumerate}


\end{document}